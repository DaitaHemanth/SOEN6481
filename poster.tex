

\documentclass[final]{beamer}

% ====================
% Packages
% ====================

\usepackage[T1]{fontenc}
\usepackage{lmodern}
\usepackage[size=custom,width=120,height=72,scale=1.0]{beamerposter}
\usetheme{gemini}
\usecolortheme{gemini}
\usepackage{graphicx}
\usepackage{booktabs}
\usepackage{tikz}
\usepackage{pgfplots}

% ====================
% Lengths
% ====================

% If you have N columns, choose \sepwidth and \colwidth such that
% (N+1)*\sepwidth + N*\colwidth = \paperwidth
\newlength{\sepwidth}
\newlength{\colwidth}
\setlength{\sepwidth}{0.025\paperwidth}
\setlength{\colwidth}{0.3\paperwidth}

\newcommand{\separatorcolumn}{\begin{column}{\sepwidth}\end{column}}

% ====================
% Title
% ====================

\title{SOEN 6481 Software Requirement Specification \newline
(Eternity Numbers)}

\author{Venkata Sai Parthasarathi Hemanth Daita }

\institute[]{ \samelineand}



\begin{document}

\begin{frame}[t]
\begin{columns}[t]
\separatorcolumn

\begin{column}{\colwidth}

  \begin{block}{Project Introduction}

    The project is all about designing a calculator which can calculate Liouville's constant and perform basic mathematical operations.\newline\newline

  \end{block}
\newline
\newline
\newline
  \begin{alertblock}{Calculator And Constant's Overview}
  The terms in the continued fraction expansion of every Liouville number are unbounded; using a counting argument, one can then show that there must be uncountably many transcendental numbers which are not Liouville.\newline
    Liouville's constant is a decimal fraction with a 1 in each decimal place corresponding to a factorial n! and zeros everywhere else.\newline
    Liouville's constant is evaluated using a key
    placed on a custom designed calculator.\newline
    \begin{itemize}
      \item \textbf A key/button is placed at the bottom of the calculator. 
      \item \textbf Constant value is displayed by entering a number and clicking the liouville button placed at the bottom of the calculator
      \item \textbf The constant value is dispalyed upto 5040(7!) decimal points.
      \item \textbf No rational number is transcendental and all real transcendental numbers are irrational.
    \end{itemize}

  \end{alertblock}
  \newline
  \newline

  \begin{block}{Interesting Points About Constant}


    \begin{itemize}
      \item \textbf Liouville's constant is a decimal fraction with a 1 in each decimal place corresponding to a factorial n! and zeros everywhere else.
      \item \textbf Liouville number was the first decimal constant to be proven transcendental.
      \item \textbf The gap between two 1's is gradually increased as the number gets larger.
      \item \textbf All liouville numbers are transcendental but not vice versa.
    \end{itemize}

  However, a mathematician named Cantor later proved that almost all real numbers are transcendental.

  \end{block}
  
  \begin{alertblock}{Applications}
    There are no practical uses of Liouville's number.
    However, by establishing that a given number is a Liouville number provides a useful tool for proving a given number is transcendental.

    \begin{itemize}
      \item \textbf A button has been provided in the calculator which on entering displays if the entered number is a transcendental or not. 
    \end{itemize}

  \end{alertblock}

\end{column}

\separatorcolumn

\begin{column}{\colwidth}

  \begin{block}{Challenges Faced}

   A number of challenges were encountered while generating the liouville's number
    \begin{enumerate}
      \item \textbf Since Liouville number generated has many decimal places(for n>=3), I faced challenge in holding the value in a data type. 
      \item \textbf Initially, I faced a challenge in choosing the User Interface to use for implementing the calculator.
      \item \textbf Since, there are no practical uses for the number, I found it difficult to include an application for the constant in the calculator.
      \item \textbf I found it difficult to calculate the constants' value without using the Java internal math functions.
      \item \textbf I found it challenging to find someone who has knowledge about my constant.
    \end{enumerate}

  \end{block}

  \begin{block}{Project Outcome}

   The following is the custom00 designed calculator outcome of the project.
\begin{figure}[!http]
    \centering
    \fbox{\includegraphics[width=\textwidth]{images/calculator.PNG}}
\end{figure}

  \end{block}

  
\end{column}

\separatorcolumn

\begin{column}{\colwidth}

  \begin{alertblock}{Critical Decisions Made}

\item \textbf Since generating the liouville number for n value greater than 3 generates a min. of 24 decimal points(120 decimal points for n=5 and so on), and data type can usually hold (double data type) upto 16-17 decimal points, a critical decision had to be made for generating the liouville number for n value greater than 3. After consulting the other team members, I found a solution in holding the values in a data type. Big decimal data type holds values upto 7! decimals (5040 decimal points).
 \item \textbf A critical decision had to be made regarding the User interface to be used in implementing the calculator. I have chosen using jframes for implementing the calculator rather than command line interface. 
 \item \textbf found it difficult to constant value generation  without using the internal math functions as it had some complex computations. Eventually found a way to implement it without using math functions.  
  \end{alertblock}

  \begin{block}{Lessons Learned}

    \begin{enumerate}
      \item \textbf  I have learned and gained knowledge about transcendental numbers, in particular Liouville number. 
      \item \textbf I have learned that a transcendental number is a real number that is not the solution of any single-variable polynomial equation whose coefficients are all integers .
      \item \textbf I have learned that almost all real numbers are transcendental.
       \item \textbf I have found various ways to obtain the critical information about the constant via surveys, persona, interviews etc.
       \item \textbf I found a way to shortlist the main user stories which can have a great impact on the project based on the following factors such as estimation points, constraints, priority etc.
        \item \textbf Initially, I designed without including any application for my constant. On review and advice by one of my team members, I have included an application for my number.
        \item \textbf  There were few computational errors. On reviewing and testing by one of the team members, I was able to find the errors and rectify early.
    \end{enumerate}

  \end{block}
  \begin{alertblock}{Conclusion}

    Implemented a custom designed calculator which calculates and displays the liouville's constant. I have gained a fair knowledge about irrational numbers, transcendental numbers and my constant.

  \end{alertblock}

\end{column}

\separatorcolumn
\end{columns}
\end{frame}

\end{document}
